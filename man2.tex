\documentclass[a4paper,10pt]{article}
\usepackage{anysize}
\usepackage{amsmath}
\usepackage{amssymb}
\usepackage{graphicx}
\usepackage[left=0.6in, right=0.6in, top=0.6in, bottom=0.75in, includefoot, headheight=13.6pt]{geometry}
\usepackage{color,graphicx}
\usepackage{verbatim}
\usepackage{hyperref}

\hypersetup{
    bookmarks=true,         % show bookmarks bar?
    unicode=false,          % non-Latin characters in Acrobat's bookmarks
    pdftoolbar=true,        % show Acrobat's toolbar?
    pdfmenubar=true,        % show Acrobat's menu?
    pdffitwindow=true,      % page fit to window when opened
    pdftitle={Resume},    % title
    pdfauthor={MANU T S},     % author
    pdfsubject={Resume for placements},   % subject of the document
    colorlinks=true,       % false: boxed links; true: colored links
    linkcolor=magenta,        % color of internal links
    citecolor=blue,        % color of links to bibliography
    filecolor=magenta,      % color of file links
    urlcolor=cyan           % color of external links
}

\definecolor{titleColor}{rgb}{0.85, 0.85, 0.85}

\begin{document}
{\qquad \\ \\ \\ \\ \\ \\ \\ \\ \\ \\}

%%%%%%%%%%___AREAS OF INTEREST____%%%%%%%%
\colorbox{titleColor}{\parbox{6.5in}{\textbf{AREAS OF INTEREST}}}

 \begin{tabular}{p{3.5in}p{5in}p{3.5in}}
    $\circ$ Communication Systems \\
    $\circ$ Digital Signal Processing \\
    $\circ$ Information Theory \\
    $\circ$ Wireless Communication \\
    $\circ$ Error Correcting Codes \\
    $\circ$ Software Defined Radio
  \end{tabular}

%%%%%%%%%%___TECHNICAL SKILLS____%%%%%%%%
 \colorbox{titleColor}{\parbox{6.5in}{\textbf{TECHNICAL SKILLS}}}
 
 \begin{itemize}
 \setlength{\itemsep}{1pt}
 \item \textbf{{Operating Systems:}} Linux, Windows.
 \item \textbf{{Softwares Skills:}} GNU Radio, Scipy, Scilab and \LaTeX
 \item \textbf{{Programming Skills:}} C, C++ and Python
 \item \textbf{{Software Defined Radio Platforms:}} USRP, RTL SDR
 \end{itemize}

%%%%%%%%%%% ____PROJECTS AND SEMINARS___ %%%%%%%%%%%
 \colorbox{titleColor}{\parbox{6.5in}{\textbf{PROJECTS AND SEMINARS}}}

 \begin{itemize}
 \setlength{\itemsep}{1pt}
 \item \textbf{Google Summer of Code, 2013:} \textbf{LDPC codes and more FEC in GNU Radio}  \textbf \emph{(June 2013-Present)}\\
        {\textbf{Mentor} - Dr.-Ing. Jens Elsner, CEL, KIT.   }          %\qquad \textbf{Software Used}: MATLAB}

        $\circ$ The project is to develop generic encoders and decoders for LDPC Codes in GNU Radio. \\
        $\circ$ As a part of this project we also improve encoders and decoders for BCH and RS codes. \\
        $\circ$ Algorithms for obtaining LDPC Codes are also implemented. \\
        $\circ$ Block for belief-propagation decoder is implemented. \\
        $\circ$ Block for back-substitution encoding is implemented. \\
        $\circ$ Project is open-source and is available at \url{https://github.com/manuts/ldpc}

        \item \textbf{M.Tech. Project:} \textbf{Application of LDPC codes to Multiuser Communications} \textbf \emph{(July 2013-Present)} \\
        {\textbf{Guide} - Prof. Sibiraj B Pillai, IIT Bombay.}            %\qquad \textbf{Software Used}: MATLAB}
        
$\circ$ LDPC codes are characterized by sparse parity check matrices \\ 
$\circ$ The project is to study LDPC encoding and decoding algorithms and extend them to multiuser scenario. \\
$\circ$ Belief-propagation decoder and back-substitution encoder blocks are developed.

        
\item \textbf{DARPA Spectrum Challenge:} \textbf \emph{(March. 2013-Present)}

$\circ$ During the hurdles and wild-card tournament we designed a transmitter-receiver pair, for two scenarios. \\
$\circ$ A pair of nodes competing against another to communicate a file in shortest time is designed. \\
$\circ$ A pair of nodes cooperating with two other pairs is designed. \\
$\circ$ All the pairs use same the same 5MHz frequency band. \\
$\circ$ Single-handedly pushed the team through wild-card tournament. \\
$\circ$ In the wild-card tournament our team surpassed teams from top universities and industries.

 \item \textbf{M.Tech. Seminar:} \textbf{Resource allocation in Wireless Networks}  \textbf \emph{(Nov. 2011)}\\
        {\textbf{Guide} - Prof. Sibiraj B Pillai, IIT Bombay.   }

      $\circ$ We studied optimal power allocation in flat-fading multiple access channels. \\
      $\circ$ Schemes achieving rate tuples under information theoretic setup were studied.
             
 \item \textbf{DSP Course Project:} \textbf{Design of Digital Filters} \textbf \emph{(Nov. 2011)}\\
	{\textbf{Instructor} - Prof. Vikram M Gadre, IIT Bombay. }
	
	$\circ$ Designed FIR and IIR filters. \\
	$\circ$ Designed filters under band-stop, band-pass and low-pass responses. \\
	$\circ$ Filters were designed under chebyschev and buttorworth approximations.
	
 \item \textbf{DSP Course Project:} \textbf{Localization of audio source} \textbf \emph{(Nov. 2011)}\\
	{\textbf{Instructor} - Prof. Vikram M Gadre, IIT Bombay. }
	
	$\circ$ In this project we designed a system to locate an audio source. \\
	$\circ$ The delay between audio signals captured from two mics is used to locate audio source.

  \item \textbf{{B. Tech. Main Project}: Study of Orthogonal Frequency Division Multiplexing}  \textbf \emph{(Jan. - June 2010)}\\
        {\textbf{Guide} - Prof. P Harikrishna Prasad, NIT, Warangal. }
        
$\circ$ The goal of the project was to study OFDM systems. \\
$\circ$ Matlab simulation of an OFDM system was done as part of this project.

 \end{itemize}

%%%%%%%%%%___COURSE WORKS___%%%%%%%%%%%
 \colorbox{titleColor}{\parbox{6.5in}{\textbf{COURSE WORK}}}

 \begin{tabular}{p{3.5in}p{5in}p{3.5in}}
     $\circ$ Communication Systems \\
     $\circ$ Error Correcting Codes \\
    $\circ$ Information Theory \\
    $\circ$ Digital Message Transmission \\
    $\circ$ Digital Signal Processing \\
    $\circ$ Wireless Communication \\
    $\circ$ Statistical Signal Analysis \\
    $\circ$ Applications of Linear Algebra \\
    $\circ$ Markov chains and Queuing System \\
    $\circ$ Optimization Techniques \\
\end{tabular}

%%%%%%%%%%___ACHIEVEMENTS____%%%%%%%%
 \colorbox{titleColor}{\parbox{6.5in}{\textbf{ACHIEVEMENTS}}}

 \begin{itemize}
  \item Our team secured a spot in the finals of \textbf{DARPA Spectrum Challenge}.
  \item Our team won first prize in Junk-Yard wars in Technozion 2010.
  \item Best outgoing student of the year 2004 in JNV Malampuzha.
  \end{itemize}

%%%%%%%%%%% ____WORK EXPERIENCE___ %%%%%%%%%%%
 \colorbox{titleColor}{\parbox{6.5in}{\textbf{WORK EXPERIENCE}}}

 \begin{itemize}
 \setlength{\itemsep}{1pt}
 \item \textbf{Graduate Engineering Trainee ( C \& I ), Adani Power Limited}  \textbf \emph{(July 2010-January 2011)}\\ 
 $\circ$ Obtained training on various functional units like boiler, turbine, water-treatment etc in a power plant.\\
 $\circ$ Exposure to super-critical thermal power plant. \\
 $\circ$ Obtained two weeks of training on soft-skills during this period.
 
   \item \textbf{Industrial Training:} \textbf{BSNL Kerala Circle}  \textbf \emph{(May 2008 - June 2008)}        

  $\circ$ Various aspects of GSM Architecture were studied. \\
  $\circ$ Training on OMC Radio, OMC Switch, Radio Planning and BSS were obtained. \\
  $\circ$ RAN drive tests were done. \\
  $\circ$ Realtime traffic analysis of telecom networks were conducted.

\end{itemize}

  %%%%%%%%%%___POSITION OF RESPONSIBILITY___%%%%%%%%%%%
 \colorbox{titleColor}{\parbox{6.5in}{\textbf{POSITIONS OF RESPONSIBILITY}}}

 \begin{itemize}
 \setlength{\itemsep}{1pt}
 \item \textbf{Teaching Assistant} \\
I was assigned the duty of System Administrator in the following lab courses.\\
$\circ$ EE-318, Electronics Design Lab - I, Spring 2012. \\
$\circ$ EE-712, Embedded System Design Lab, Spring 2012 . \\ 
$\circ$ EE-389, Electronics Design Lab - II, Autumn 2012.  \\
$\circ$ EE-616, Electronics System Design Lab, Autumn 2012. \\
\qquad \\
As the System Administrator I did the following things. \\
$\circ$ Configure and maintain all the computers in used in the labs. \\
$\circ$ Configure and maintain four server grade machines in the labs. \\
$\circ$ Configured Apache server, FTP server, Samba server and DHCP server. \\
$\circ$ Wrote a bash script to automate user maintenance.
 \end{itemize}

%%%%%%%%%%___HOBBY___%%%%%%%%%%%
 \colorbox{titleColor}{\parbox{6.5in}{\textbf{HOBBIES}}}

  \begin{itemize}
 \setlength{\itemsep}{1pt}
    \item Reading, coding, music and movies.
  \end{itemize}



\end{document}
